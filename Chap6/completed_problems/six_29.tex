\documentclass{article}

\usepackage{amsmath}
\usepackage{amsfonts}
\usepackage[margin = .5in]{geometry}
\usepackage{enumerate}

\begin{document}

\noindent \textbf{Exercise Number: 6.29}  %% FILL THIS IN

\medskip 

We have the following assumptions:

\begin{itemize}

\item Memory is byte-addressable, that is to say, addresses represent a location containing
a byte of memory

\item Memory accesses are 1-byte words, meaning the block of memory access, sent through the
memory bus, and loaded into the cache at once is a byte of memory

\item Address are 12 bit wide ($m  = 12$) 

\item Two lines per set ($E=2$), 4-byte block size ($B = 4$), and four sets or $S = 4$.

\end{itemize}

Below is a table of the contents of the cache at the beginning of the problem. Note the bytes 
are represented with 2 hex digits. 

\bigskip

\begin{center}
  \begin{tabular}{*{7}{c}}
   Set Index  &   Tag   &   Valid   &  Byte0 &   Byte1  &   Byte2   &   Byte3   \\ \hline

    0         &   00    &   1       &  40    &    41    &   42      &   43  \\

    0         &   83    &   1       &  FE    &   97     &   CC      &   D0  \\

    1         &   00    &   1       &  44    &   45     &   46      &   47  \\

    1         &   83    &   0       & --     &   --     &   --      &   --  \\

    2         &   00    &   1      &  48     &   49     &  4A     &   4B    \\

    2         &   40    &   0      &  --     &   --     &  --      &   --   \\

    3         &   FF    &   1      &  9A   &   C0    &  03     &   FF       \\

    3         &   00    &   0       &  --    &  --  &  --      &   --       \\

  \end{tabular}
\end{center}

\bigskip


\begin{enumerate}[A.]

\item We fill in the given diagram (which erroneously presents 13 bits: that is corrected here)
where t represents a tag bit, s a set bit, and b a block bit. The lower row represents the index of the bit

\begin{center}
  \begin{tabular}{*{12}{|c} |}
    
    \hline
    t & t & t & t & t & t & t  & t  & s  & s & b & b     \\ \hline
    11 & 10 & 9 & 8 & 7 & 6 & 5 & 4 & 3 & 2 & 1 & 0

  \end{tabular}
\end{center}

In other words, there are 8 tag bits, 2 set bits, and 2 block offset bits. 

\item We fill in the given table as presented, and then explain the logic behind the solution
and the above diagram. 

\begin{center}
  \begin{tabular}{*{7}{c}}
   Operation  &   Address  &   Hit?   &  Read Value (or unknown)  \\ \hline

    Read      &   0x834    &   No     &  Unknown \\

    Write     &   0x836    &   Yes    &   Unknown         \\

    Read     &   0xFFD    &  Yes   &   9A      \\ 

  \end{tabular}
\end{center}
 
\begin{enumerate}

\item The first instruction has tag bits 83, or set + block bits represented by hex 4. This 
translates to 01 set bits (or set 1) and 00 block bits (or block 0). The tag 83 is in the 
second line of the first set, but the valid bit in this line is 0 so the cache misses. 

\item The second intruction ends with a 6 which has binary representation 0110. This means 
the set referenced is 1 and the block offset is 2. Since this is in the line that previously
missed, the cache is now warmed up and we have a hit. This is a write operation so the CPU
isn't seeking to retrieve anything from memory: this implies that we can only put down that the value is unknown.  

\item D has binary representation 1101. This means that set 3 is referenced here, tag FF. 
Looking in the table we can see that this is a hit. The request retrieve block offset 01, 
or byte0 = 9A.  

\end{enumerate}

\end{enumerate}

\bigskip

\noindent \textbf{Discussion.} Not much to see here...fairly routine operation. The point is
to give some intuition for how cache mappings work: two of the miss operations presented were
within the same line, which makes sense because their addresses were so similar. The cache then 
warms up and the second operation is a hit.  

\end{document} 