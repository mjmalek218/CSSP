\documentclass{article}

\usepackage{amsmath}
\usepackage{amsfonts}
\usepackage[margin = .5in]{geometry}

\begin{document}

\noindent \textbf{Exercise Number: 6.2.5}  %% FILL THIS IN

\medskip 

\noindent In this problem we are required to fill in the given table related to
cache memories/specs. It is quite simple once the relation $C = S \times B \times E$ is recalled, as well as $t = m - s - b$.

\bigskip

\begin{center}
  \begin{tabular}{l*{7}{c}r}
    Cache &   $m$   &   $C$   &  $B$ &   $E$  &   $S$   &   $t$   &  $s$  &   $b$   \\ \hline

    1     &   32    &   1024  &  4   &    4   &   64    &   24    &   6   &   2  \\

    2     &   32    &   1024  &  4   &   256  &  1      &   30    &   0   &  2    \\

    3     &   32    &   1024  &  8   &   1    &  128    &   22    &   7   &  3    \\

    4     &   32    &   1024  &  8    &   128 &  1      &   29    &   0   &  3    \\

    5     &   32    &   1024  &  32   &   1   &  32     &   22    &   5   &  5     \\

    6     &   32    &   1024  &  32   &  4    &  8      &   24   &   3    &  5    \\

  \end{tabular}
\end{center}

\bigskip

\bigskip

\noindent \textbf{Discussion.}

A few things to recognize here, primarily the general pattern/cost of increasing one aspect of the 
cache and its influence on another aspects.

In all cases above, the cache is a fixed size 1024 bytes. Caches 2 and 4 are fully associative
(like virtual memory) whereas caches 3 and 5 are direct mapped (closer to smaller caches).

Also in general there are quite a few tag bits. The larger the number of tag bits, the greater
the disparity between the size of the cache and the size of main memory (or whatever the lower
level is that it's calling from). 

\end{document} 