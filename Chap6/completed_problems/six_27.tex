\documentclass{article}

\usepackage{amsmath}
\usepackage{amsfonts}
\usepackage[margin = .5in]{geometry}
\usepackage{enumerate}

\begin{document}

\noindent \textbf{Exercise Number: 6.2.7}  %% FILL THIS IN

\medskip 

\noindent In this problem, we are asked to list specific memory addresses that are 13 bits long, 
so naturally if we represent memory addresses in hex there will be 4 hexadecimal digits
required, with the last one only varying between 0 and 1--regardless of other constraints
imposed by the further specifications of this problem.

There are also 2 block bits and 3 set bits.

\begin{enumerate}[A.]

\item In our first problem, let $x$ denote an arbitrary bit: 0 or 1. If we know $S = 1$, then the
set bits must be 001, and so we have the following general format for any adress satisfying our
constraints: 

\begin{center}
  \begin{tabular}{*{13}{|c} |}
    
    \hline
    x  & x & x & x & x & x & x & x  & 0  & 0  & 1 & x & x     \\ \hline
  \end{tabular}
\end{center}

Since we partition these into 4 to digits each, starting from the left, and the $x$'s are allowed
to vary freely, it follows that of the five hex digits we could possibly have representing this
address:

\begin{enumerate}

\item The first will be any binary number the first two bits could possibly be, plus $4$, so
4, 5, 6, 7.

\item Second one must have first bit 0, so it is all even numbers from 0 to 15

\item Third is allowed to vary freely

\item Fourth can only be 0 or 1, since it is only one bit. 

\end{enumerate}

\item Essentially we have the same structure here, except it's in the format:

\begin{center}
  \begin{tabular}{*{13}{|c} |}
    
    \hline
    x  & x & x & x & x & x & x & x  & 1  & 1  & 0 & x & x     \\ \hline
  \end{tabular}
\end{center}


\begin{enumerate}

\item The first will be any binary number the first two bits could possibly be, plus $8$, so
8, 9, 10, 11.

\item Second one must have first bit 1, so it is all odd numbers from 0 to 15

\item Third is allowed to vary freely

\item Fourth can only be 0 or 1, since it is only one bit. 

\end{enumerate}


\end{enumerate}

\noindent \textbf{Discussion.} Not much to discuss here...just kind of going through the motions.


\end{document} 